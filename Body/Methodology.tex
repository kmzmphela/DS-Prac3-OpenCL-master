\section{Methodology}
This section describes the methods used in these experiments.

\subsection{Device Information}

\subsubsection{CPU}
\begin{itemize}
	\item Device Name: Intel(R) Xeon(R) CPU E5-2686 v4 @ 2.30GHz
	\item Available Compute Units: 4
	\item Available work Units: [8192, 8192, 8192]
	\item Clockrate: 2300
	\item Available Alloc Memory: 16097245184\\
\end{itemize}


\subsubsection{GPU}
\begin{itemize}
	\item Device Name: Tesla K80
	\item Available Compute Units: 13
	\item Available work Units: [1024, 1024, 64]
	\item Clockrate: 823
	\item Available Alloc Memory: 2998894592
\end{itemize}

\subsection{Implementation}
Also mention the implementation source code:


\begin{OpenCL_float}{OpenCL code to obtain device information}{OpenCL_Device_Info}
name_properties = {
	"Device Name":pyopencl.device_info.NAME,
}

processing_properties = {
	"Available Compute Units": pyopencl.device_info.MAX_COMPUTE_UNITS,
	"Available work Units": pyopencl.device_info.MAX_WORK_ITEM_SIZES,
	"Clockrate": pyopencl.device_info.MAX_CLOCK_FREQUENCY
}

memory_properties = {
	"Available Alloc Memory": pyopencl.device_info.MAX_MEM_ALLOC_SIZE
}

for device in (nvidia_device,intel_device):
	#print out all of the device name properties, except the device type
	for property_name in sorted(name_properties.keys() - {"Device Type"}):
		property_string_args =\\
		(property_name,device.get_info(name_properties[property_name]))
		print("%s: %s"%property_string_args)

	#print out all of the processing properties
	for property_name in sorted(processing_properties.keys()):
		property_string_args =\\
		(property_name,device.get_info(processing_properties[property_name]))
		print("%s: %s"%property_string_args)
	
	#print out all of the memory properties
	for property_name in sorted(memory_properties.keys()):
		property_string_args =\\
		(property_name,device.get_info(memory_properties[property_name]))
		print("%s: %s"%property_string_args)
	
	print("\n")
\end{OpenCL_float}

\begin{Python}{OpenCL run_cpu_program}{cpu_sum}
def run_cpu_program():
	#copying data onto CPU
	pyopencl.enqueue_copy(intel_queue,
	src=a,
	dest=a_intel_buffer)
	pyopencl.enqueue_copy(intel_queue,
	src=b,
	dest=b_intel_buffer)
	
	#running program
	kernel_arguments = (a_intel_buffer,b_intel_buffer,c_intel_buffer) 
	intel_program.sum(intel_queue,
	a.shape, #global size
	None, #local size
	*kernel_arguments)
	
	#copying data off CPU
	copy_off_event = pyopencl.enqueue_copy(intel_queue,
	src=c_intel_buffer,
	dest=c)
	copy_off_event.wait()
\end{Python}


\begin{Python}{OpenCL run_gpu_program}{gpu_sum}
def run_gpu_program():
	#copying data onto GPU
	pyopencl.enqueue_copy(nvidia_queue,
	src=a,
	dest=a_nvidia_buffer)
	pyopencl.enqueue_copy(nvidia_queue,
	src=b,
	dest=b_nvidia_buffer)
	
	#running program
	kernel_arguments = (a_nvidia_buffer,b_nvidia_buffer,c_nvidia_buffer) 
	nvidia_program.sum(nvidia_queue,
	a.shape, #global size
	None, #local size
	*kernel_arguments)
	
	#copying data off GPU
	copy_off_event = pyopencl.enqueue_copy(nvidia_queue,
	src=c_nvidia_buffer,
	dest=c)
	copy_off_event.wait()
\end{Python}

\begin{Cpp}{C Kernel factor_count}{factor_count}
#define data_t int

// Factor Count
// X - N x M matrix
// factors - N x M --
kernel void factor_count(global data_t *X,
global data_t *factors,
int N,
int M)
{
	int row = get_global_id(0);
	int col = get_global_id(1);
	
	int offset = row*M; //start of row of X and factor
	
	//Calculation loop
	data_t sum = 0; //private variable used to cache result
	for(int i=0;i<N;++i) {
		for(int j=0;j<M;++j) {
			int nf = 0;
			for(int k=2;k<101;++k) {
				if( (X[offset + j + i*M] % k) == 0)
				++nf; 
			}
			factors[offset + j + i*M] = nf;
		}
	}
}
	
\end{Cpp}


\begin{Python}{Python factor_count program}{factor_count}
def opencl_factor_count(data_arrays,buffers,queue,kernel,N=100,M=100):
	a = data_arrays
	a_buffer,c_buffer = buffers
	N,M = numpy.int32(M),numpy.int32(N) #converting to the exepcted type for OpenCL
	
	#copying data onto device
	copyon_events = []
	
	copyon_events += [pyopencl.enqueue_copy(queue,
	src=a,
	dest=a_buffer)]
	
	#running program
	kernel_event = kernel(queue,
	(N,N), #global size
	None, #local size 
	a_buffer,c_buffer,N,M, #Kernel Arguments
	wait_for = copyon_events)
	
	
	#copying data off device
	c = numpy.empty((N,N),dtype=dt)
	copyoff_event = pyopencl.enqueue_copy(queue,
	src = c_buffer,
	dest = c,
	wait_for = [kernel_event])
	copyoff_event.wait()
	
	return c

data_arrays = create_arrays()

get_ipython().magic('timeit opencl_factor_count(data_arrays,nvidia_buffers,\\
nvidia_queue,nvidia_program.factor_count)')
get_ipython().magic('timeit opencl_factor_count(data_arrays,intel_buffers,\\
intel_queue,intel_program.factor_count)')

\end{Python}
Only list what is relevant.  Don't give too much detail - just enough to show what you've done.  This template supports the following languages:

  
\subsection{Experiment Procedure}
Furthermore, include detail relating to the experiment itself: what did you do, in what order was this done, why was this done, etc.  What are you trying to prove / disprove?